\documentclass[a4paper,12pt]{report}
\usepackage[english]{babel}
\usepackage{graphicx}
\usepackage{placeins}
\usepackage{hyperref}
\usepackage{supertabular}
\usepackage{listings}
\usepackage{color}
\lstset{% general command to set parameter(s)
basicstyle=\footnotesize,
keywordstyle=\color{black}\bfseries\underbar,
identifierstyle=\color{black}\bfseries\underbar,
stringstyle=\ttfamily,
}




\title{FoLiA: Format for Linguistic Annotation \\ \small v0.3}
\author{Maarten van Gompel \\ ILK Research Group \\ Tilburg center for Cognition and Communication \\ Tilburg University }

\parindent=0pt %no paragraph indentation
\parskip=12pt %paragraph skip


\newenvironment{devnotes}
{
\begin{center}
    \begin{tabular}[h!]{|p{0.8\textwidth}|}
    \hline
    {\bf Development Notes}\\\hline}
{   \\\hline
    \end{tabular}
\end{center}}

\begin{document}
\sffamily

\maketitle
\tableofcontents


\chapter{Introduction}

FoLiA is a Format for Linguistic Annotation, derived from the D-Coi format\cite{DCOI} developed as part of the D-Coi project by project partner at Polderland Language and Speech Technologies B.V. The D-Coi format was designed for use by the DCOI corpus, as well as by its successor; the SoNaR corpus \cite{Oostdijk+08}. Though being rooted in the D-Coi format, the FoLiA format goes a lot further and introduces a rich generalised framework for linguistic annotation. FoLiA is developed at the ILK research group, Tilburg University, and proposed as a CLARIN standard in the TTNWW project.

FoLiA is an XML-based\cite{XML} annotation format, suitable for representing written language resources such as corpora. Its goal is to unify a variety of linguistic annotations in one single rich format, without committing to any particular standard annotation set. Instead, it seeks to accommodate any desired system or tagset, and so offer maximum flexibility. This makes FoLiA language independent. Due to its generalised set up, it is easy to extent the FoLiA format to suit your custom needs for linguistic annotation.

XML is an inherently hierarchic format. FoLiA does justice to this by maximally utilising a hierarchic, inline, setup. We inherit from the D-Coi format, which posits to be loosely based on a minimal subset of TEI\cite{TEI}. Because of the introduction of a new and broader paradigm, FoLiA is \emph{not} backwards-compatible with D-Coi, i.e. validators for D-Coi will not accept FoLiA XML. It is however easy to convert FoLiA to less complex or verbose formats such as the D-Coi format, or plain-text. Converters will be provided. This may entail some loss of information if the simpler format has no provisions for particular types of information specified in the FoLiA format. 


In contrast to the D-Coi format, the FoLiA format introduces annotation layers separate from the token-based skeleton structure, to capture structured linguistic annotations such as syntactic parses. This is to provide FoLiA with the necessary flexibility and extensibility. Inspiration for this was in part obtained from the Kyoto Annotation Format \cite{KYOTO}. 

The FoLiA format features the following:

\begin{itemize}
\item Open-source
\item XML-based, validation against XML schema.
\item Full Unicode support; UTF-8 encoded.
\item Document structure consists of divisions, paragraphs, sentences and words/tokens.
\item Can encode both tokenised as well as untokenised text + partial reconstructability of untokenised form even after tokenisation.
\item Support for crude token categories (word, punctuation, number, etc)
\item Explicit support for encoding quotations
\item Provenance support for all linguistic annotations: annotator, type (automatic or manual), time.
\item Support for alternative annotations, optionally with associated confidence values.
\item Adaptable to different tag-sets.
\item Agnostic with regard to metadata. CMDI is recommended, but alternatives like IMDI can also be used.
\end{itemize}

It supports the following linguistic annotations:

\begin{itemize}
\item Part-of-Speech tags (with features)
\item Lemmatisation
\item Spelling corrections on both a tokenised as well as an untokenised level
\item Lexical semantic sense annotation (to be used in DutchSemCor)
\item Named Entities / Multi-word units
\item Syntactic Parses
\item Dependency Relations
\item Chunking
\end{itemize}

In later stages, the following may be added:

\begin{itemize}
\item Morphological Analysis
\item Semantic Role Labelling
\item Co-reference
\item Topic Segmentation
\item Authorship Attribution
\end{itemize}


FoLiA support will be incorporated directly into the following ILK sofware:

\begin{itemize} 
\item ucto - A tokeniser which can directly output FoLiA XML 
\item Frog - A PoS-tagger/lemmatiser/parser suite (the successor of Tadpole), will eventually support reading and writing FoLIA.
\item CLAM - Computational Linguistics Application Mediator, will eventually have viewers for the FoLiA format.
\item PyNLPl - Python Natural Language Processing Library, will come with libraries for parsing FoLiA
\item libfolia - C++ library for parsing FoLiA
\end{itemize}

FoLiA will be used in the DutchSemCor project, and will also be proposed for consideration in the SoNaR project.

To clearly understand this documentation, note that when we speak of ``elements'' or ``attributes'', we refer to XML notation, i.e. XML elements and XML attributes.

\chapter{Format}

\section{Global Structure}

In FoLiA, each document/text is represented by one XML file. The basic structure of such a FoLiA document is as follows and should always be UTF-8 encoded. An elaborate XSLT stylesheet will be provided in order to be able to instantly view FoLiA documents in any modern web browser.

\begin{lstlisting}[language=xml]
<?xml version="1.0" encoding="utf-8"?>
<?xml-stylesheet type="text/xsl" href="http://ilk.uvt.nl/FoLiA/FoLiA.xsl"?>
<FoLiA xmlns="http://ilk.uvt.nl/FoLiA"
  xmlns:xsi="http://www.w3.org/2001/XMLSchema-instance" xml:id="example">
  <metadata>
      <annotations>
          ...
      </annotations>    
      <!-- (Here CMDI or IMDI metadata can be inserted) -->
  </metadata>
  <text xml:id="example.text">
     ...
  </text>
</FoLiA>  
\end{lstlisting}



\section{Identifiers}

Many elements in the FoLiA format specify an identifier by which the element is uniquely identifiable. This makes referring to any part of a FoLiA document easy and follows the lead of the D-Coi format. The identifiers are constructed in the same way as in the D-Coi format, thus retaining full compatibility if a D-Coi document is converted to FoLiA, any external references to any entity in these documents will remain intact.

Identifiers in D-Coi and FoLiA are cumulative and are usually formed by appending the elements name, a period, and a sequence number, to the identifier of a parent element higher in the hierarchy.

The base of all identifiers is that of the document itself, as encoded in \texttt{xml:id} attribute of the root \texttt{FoLiA} element. This is a unique ID by which the document is identifiable. We choose the identifier \emph{example} for all of the examples in this manual. By convention, the XML file should then ideally be named: \texttt{example.xml}.

Identifiers are very important and used throughout the FoLiA format. They enable external resources and database to easily point to a specific part of the document or its annotation. FoLiA has been set up in such a way that \emph{identifiers should never ever change}. Once an identifier is assigned, it should never change, re-numbering is strictly prohibited unless you intentionally want to create a new resource and break compatibility with the old one.


\section{Basic Structural Elements}

Basic structural elements occur within the \texttt{text} element. These are the most basic ones:

\begin{itemize}
\item \texttt{p} - Paragraph
\item \texttt{s} - Sentence
\item \texttt{w} - Word (token)
\end{itemize}

These are typically nested, the word elements cover the actual tokens. This is the most basic level of annotation; tokenisation. Let's take a look at an example, we have the following text:

\begin{verbatim}
This is a paragraph containing only one sentence.

This is the second paragraph. This one has two sentences.
\end{verbatim}

In FoLiA XML, this will appear as follows after tokenisation. Some parts have been omitted for the sake of brevity:


\begin{lstlisting}[language=xml]
 <p xml:id="example.p.1">
    <s xml:id="example.p.1.s.1">        
        <w xml:id="example.p.1.s.1.w.1"><t>This</t></w>
        <w xml:id="example.p.1.s.1.w.2"><t>is</t></w>
        ...
        <w xml:id="example.p.1.s.1.w.8" space="no"><t>sentence</t></w>
        <w xml:id="example.p.1.s.1.w.9"><t>.</t></w>
    </s>
 </p>
 <p xml:id="example.p.2">
    <s xml:id="example.p.2.s.1">
        <w xml:id="example.p.2.s.1.w.1"><t>This</t></w>
        <w xml:id="example.p.2.s.1.w.2"><t>is</t></w>    
        ..
        <w xml:id="example.p.2.s.1.w.5" space="no"><t>paragraph</t></w>    
        <w xml:id="example.p.2.s.1.w.6"><t>.</t></w>    
    </s>
    <s xml:id="example.p.2.s.2">
        <w xml:id="example.p.2.s.2.w.1"><t>This</t></w>
        <w xml:id="example.p.2.s.2.w.2"><t>one</t></w>    
        ..
        <w xml:id="example.p.2.s.2.w.5" space="no"><t>sentences</t></w>    
        <w xml:id="example.p.2.s.2.w.6"><t>.</t></w>    
    </s>
 </p>
\end{lstlisting}


The deepest content element should always contain a text element (\texttt{t}) which holds the actual textual content. The necessity of having a text element shall become apparent as you progress through this documentation; there can be many different token annotations under a word element (\texttt{w}).

FoLiA is not just a format for holding tokenised text, although tokenisation is a prerequisite for almost all kinds of annotation. However, FoLiA can also hold untokenised text, on a paragraph and/or sentence level:

\begin{lstlisting}[language=xml]
 <p xml:id="example.p.1">
    <s xml:id="example.p.1.s.1">        
        <t>This is a paragraph containing only one sentence.</t>
    </s>
 </p>
 <p xml:id="example.p.2">
    <s xml:id="example.p.2.s.1">     
        <t>This is the second paragraph.</t>
    </s>
    <s xml:id="example.p.2.s.2">     
        <t>This one has two sentences.</t>
    </s>    
 </p>
\end{lstlisting}

Higher level elements \emph{may} also contain a text element even when the deeper element does too. It is important to realise that the sentence/paragraph-level text element always contains the text \emph{prior} to tokenisation! Note also that the word element has an attribute \texttt{space}, which defaults to yes, and indicates whether the word was followed  by a space in the \emph{untokenised} original. This allows for partial reconstructibility of the sentence in its untokenised form. Reconstructing sentences is generally preferred to grabbing them from the text element at the paragraph or sentence level, as there may be corrections on the token level.

The following example shows the maximum amount of redundancy, with text elements at every level.

\begin{lstlisting}[language=xml]
 <p xml:id="example.p.1">
    <t>This is a paragraph containing only one sentence.</t>
    <s xml:id="example.p.1.s.1">        
        <t>This is a paragraph containing only one sentence.</t>
        <w xml:id="example.p.1.s.1.w.1"><t>This</t></w>
        <w xml:id="example.p.1.s.1.w.2"><t>is</t></w>
        ...
        <w xml:id="example.p.1.s.1.w.8" space="no"><t>sentence</t></w>
        <w xml:id="example.p.1.s.1.w.9"><t>.</t></w>
    </s>
 </p>
 <p xml:id="example.p.2">
    <t>This is the second paragraph. This one has two sentences.</t>
    <s xml:id="example.p.2.s.1">
        <t>This is the second paragraph.</t>
        <w xml:id="example.p.2.s.1.w.1"><t>This</t></w>
        <w xml:id="example.p.2.s.1.w.2"><t>is</t></w>    
        ..
        <w xml:id="example.p.2.s.1.w.5" space="no"><t>paragraph</t></w>    
        <w xml:id="example.p.2.s.1.w.6"><t>.</t></w>    
    </s>
    <s xml:id="example.p.2.s.2">
        <t>This one has two sentences.</t>
        <w xml:id="example.p.2.s.2.w.1"><t>This</t></w>
        <w xml:id="example.p.2.s.2.w.2"><t>one</t></w>    
        ..
        <w xml:id="example.p.2.s.2.w.5" space="no"><t>sentences</t></w>    
        <w xml:id="example.p.2.s.2.w.6"><t>.</t></w>    
    </s>
 </p>
\end{lstlisting}

The paragraph elements may be omitted if a document is described that does not distinguish paragraphs but only sentences. The identifiers of course change accordingly then. Sentences however should never be omitted; documents can never consist of tokens only!

The content element \texttt{head} is reserved for headers and captions, it behaves similarly to the paragraph element and may hold sentences.


FoLiA also explicitly support quotes, as demonstrated in the next example, which annotates the following sentence: 

\begin{verbatim}
He said: ``I do not know . I think you are right. ", and left.
\end{verbatim}

 A quote may consist of one or more sentences, but may also consist of mere tokens. The token identifiers in all cases simply follow the sequential numbering of the root sentence, not the embedded sentence.


\begin{lstlisting}[language=xml]
 <s xml:id="example.p.1.s.1">
  <w xml:id="example.p.1.s.1.w.1" class="WORD"><t>He</t></w>
  <w xml:id="example.p.1.s.1.w.2" class="WORD"><t>said</t></w>
  <w xml:id="example.p.1.s.1.w.3" class="PUNCTUATION" space="no"><t>:</t></w>
  <w xml:id="example.p.1.s.1.w.4" class="PUNCTUATION" space="no"><t>''</t></w>
  <quote xml:id="example.p.1.s.1.quote.1">
    <s xml:id="example.p.1.s.1.quote.1.s.1">
       <w xml:id="example.p.1.s.1.w.5" class="WORD"><t>I</t></w>
       <w xml:id="example.p.1.s.1.w.6" class="WORD"><t>do</t></w>
       <w xml:id="example.p.1.s.1.w.7" class="WORD"><t>not</t></w>
       <w xml:id="example.p.1.s.1.w.8" class="WORD"><t>know</t></w>
       <w xml:id="example.p.1.s.1.w.9" class="PUNCTUATION" space="no"><t>.</t></w>
    </s>
    <s xml:id="example.p.1.s.1.quote.1.s.2">
       <w xml:id="example.p.1.s.1.w.10" class="WORD"><t>I</t></w>
       <w xml:id="example.p.1.s.1.w.11" class="WORD"><t>think</t></w>
       <w xml:id="example.p.1.s.1.w.12" class="WORD"><t>you</t></w>
       <w xml:id="example.p.1.s.1.w.13" class="WORD"><t>are</t></w>
       <w xml:id="example.p.1.s.1.w.14" class="WORD"><t>right</t></w>
    </s>
  </quote>
  <w xml:id="example.p.1.s.1.w.15" class="PUNCTUATION" space="no"><t>''</t></w>
  <w xml:id="example.p.1.s.1.w.16" class="PUNCTUATION"><t>,</t></w>
  <w xml:id="example.p.1.s.1.w.17" class="WORD"><t>and</t></w>
  <w xml:id="example.p.1.s.1.w.18" class="WORD"><t>left</t></w>
  <w xml:id="example.p.1.s.1.w.19" class="PUNCTUATION" space="no"><t>.</t></w>
 </s>
\end{lstlisting}

    

\section{Paradigm \& Terminology}
\label{sec:paradigm}

The FoLiA format has a very uniform setup and its XML notation for annotation follows a generalised paradigm. We distinguish three different categories of annotation, of which the latter two apply to actual linguistic annotation:

\begin{itemize}
\item \textbf{Structural annotation} - Annotations marking global structure, such as chapters, sections, subsections, figures, list items, paragraphs, etc...
\item \textbf{Token annotation} - Annotations pertaining to one specific token. These will be elements of the token element (\texttt{w}) in inline notation. Linguistic annotations in this category are for example: part-of-speech annotation, lemma annotation, sense annotation, morphological analysis, spelling correction. Some token elements may be used on higher levels (e.g. sentence/paragraph) as well and may then be referred to as \textbf{Extended Token Annotation}
\item \textbf{Span annotation} - Annotations spanning over multiple tokens. Each type of annotation will be in a seperate \textbf{annotation layer} with offset notation. These layers are embedded on the sentence level. Examples in this category are: syntactic parses, chunking, semantic roles and named entities.
\end{itemize}

Almost all annotations are associated with what we shall call a \textbf{set}. The set determines the vocabulary, the tags or types, of the annotation. An element of such a set is referred to as a \textbf{class} from the FoLiA perspective. For example, we may have a document with Part-of-Speech annotation according to the CGN set (a tagset for Dutch part-of-speech tags). The CGN set defines main tag classes such as \emph{WW}, \emph{BW}, \emph{ADJ}, \emph{VZ}. FoLiA itself thus never commits to any tagset but leaves you to explicitly define this. You can also use multiple tagsets in the same document if so desired, even for the same type of annotation.

Any annotation element may have a \texttt{set} attribute, the value of which points to the URL hosting the file that defines the set, and a \texttt{class} attribute, which selects a class from the set.

The following example shows a simple Part-of-Speech annotation without features, but with all common attributes according to the FoLiA paradigm:

\begin{lstlisting}[language=xml]
<pos set="http://ilk.uvt.nl/folia/sets/CGN" class="WW" 
 annotator="Maarten van Gompel" annotatortype="manual"
 confidence="0.76" />
\end{lstlisting}

The example demonstrates that any annotation element can take an \texttt{annotator} attribute and an \texttt{annotatortype}. The latter is either ``manual'' for human annotators, or ``auto'' for automated systems.  The value for \texttt{annotator} is open and should be set to the name or ID of the system or human annotator that made the annotation. Last, there is a \texttt{confidence} attribute which is set to a floating point value between zero and one, the value expresses the confidence the annotator places in his annotation. None of these options are mandatory, only \texttt{class} may be mandatory for some types of annotation, such as \texttt{pos}.

More advanced aspects of the paradigm will introduced later in section \ref{sec:advparadigm}.

\begin{devnotes}
In this stage, the sets are not actually defined yet, i.e. the URLs they point to don't exist yet. But the idea is that a set always points to a URL that defines all its classes. The format for this is still to be specified however. Links to the ISOCAT Data Category Registry can later be included at that level. For now, ad-hoc sets that will later be defined will do.
\end{devnotes}


\section{Annotation Declarations}

Explicitly referring to a set and annotator for each annotation element can be cumbersome, especially in a document with a single set and a single annotator for that type particular of annotation. This problem can be solved by declaring defaults in the annotation declaration.

The annotation declaration is a mandatory part of the metadata that declares all the types of annotation that are present in the document. In addition it may define defaults such as the tagset used, a default annotator, and the type of annotator. These defaults can always be overriden at the annotation level itself, using the XML attributes \texttt{set}, \texttt{annotator} and \texttt{annotatortype}, as discussed in the previous section. None of the attributes are mandatory in the declaration, though the declarations themselves are; they declare what annotations are to be expected in the document. Having a type of annotation that is not declared is invalid. Do note that if you do not specify a set, annotator or annotator-type in either the declaration or in the annotation elements themselves, they will be left undefined. Not declaring sets is generally a bad idea.  

Annotations are declared in the \texttt{annotations} block, as shown in the following example. We here define four annotation levels.

\begin{lstlisting}[language=xml]
<annotations>
        <token-annotation set="http://ilk.uvt.nl/folia/sets/ucto-tokconfig-nl" 
          annotator="ucto" annotatortype="auto" />
        <pos-annotation set="http://ilk.uvt.nl/folia/sets/CGN" 
          annotator="Frog" annotatortype="auto" />
        <lemma-annotation annotator="Frog" annotatortype="auto" />    
        <sense-annotation set="http://ilk.uvt.nl/folia/sets/Cornetto"
         annotator="SupWSD1" annotatortype="auto" />    
</annotations>
\end{lstlisting}



\section{Structure Annotation}

\subsection{Divisions}

Within the \texttt{text} element, the structure element \texttt{div} can be used to create divisions and subdivisions. Each division may be of a particular \emph{class} pertaining to a \emph{set} defining all possible classes. Divisions and other structural units are often numbered, think for example of chapters and sections. The number, as it was in the source document, can be encoded in the \texttt{n} attribute of the structure annotation element.

Look at the following example, showing a full FoLiA document with structured divisions: 

\begin{lstlisting}[language=xml]
<?xml version="1.0" encoding="utf-8"?>
<?xml-stylesheet type="text/xsl" href="http://ilk.uvt.nl/FoLiA/FoLiA.xsl"?>
<FoLiA xmlns="http://ilk.uvt.nl/FoLiA"
  xmlns:xsi="http://www.w3.org/2001/XMLSchema-instance" xml:id="example">
  <metadata>
      <annotations>
          <div-annotation set="http://ilk.uvt.nl/folia/sets/divisions" />
      </annotations>    
      <!-- (Here CMDI or IMDI metadata can be inserted) -->
  </metadata>
  <text xml:id="example.text">
     <div class="chapter" n="1">
        <head><t>Introduction</t></head>
        <div class="section" n="1">
            <div class="subsection" n="1.1">
                <t>In the beginning....</t>
            </div>
        </div>
        ...
     </div>
  </text>
</FoLiA>  
\end{lstlisting}

If divisions are present, they need to be declared, as can be seen in the metadata. Divisions themselves are never mandatory, you can have a document without any divisions.

Divisions stem from D-Coi and are modified in FoLiA. These divisions are not mandatory, but may be used to mark extra structure. D-Coi supported the elements \texttt{div0}, \texttt{div1}, \texttt{div2}, etc.., but FoLiA only knows a single \texttt{div} element, which can be nested at will and associated with classes. Note that paragraphs, sentences and words have there own explicit tags, as seen earlier, divisions should never be used for marking these, only larger structures can be divisions!

The \texttt{head} element may be used to for the header of any division. It may hold \texttt{s} and \texttt{w} elements (not \texttt{p}).

\subsection{Gaps}

Sometimes there are parts of a document you want to skip and not annotate, but include as is. For this purpose the \texttt{gap} element should be used. Gaps may have a particular class indicating the kind of gap it is. Common omissions are for example front-matter and back-matter.

The D-Coi format pre-defined the following ``reasons'' \cite{DCOI}:

\begin{itemize}
\item frontmatter
\item backmatter
\item illegible
\item other-language
\item cancelled
\item inaudible
\item sampling
\end{itemize}

Due to the flexible nature of FoLiA, we don't predefine any classes whatsoever and leave this up to whatever set is declared. The above gives a good indication of what gaps can be used for though. 

The gap element may optionally take two elements:

\begin{enumerate}
\item \texttt{desc} - holding a substitute that may be shown to the user, describing what has been omitted.
\item \texttt{content} - The actual raw content of the omission, as it was without further annotations. This is an XML CDATA type element, excluding it from any kind of parsing.
\end{enumerate}


\begin{lstlisting}[language=xml]
  <text xml:id="example.text">
     <gap class="frontmatter" annotator="Maarten van Gompel">
        <desc>This is the cover of the book</desc>
        <content>
<![CDATA[        
    
            SHOW WHITE AND THE SEVEN DWARFS
            
            
                by the Brothers Grimm
                
                    first edition
                     
            
            Copyright(c) blah blah
]]>
        </content>
     </gap>
     <div class="chapter" n="1">
        <head><t>Introduction</t></head>
        <div class="section" n="1">
            <div class="subsection" n="1.1">
                <t>In the beginning....</t>
            </div>
        </div>
        ...
     </div>
  </text>
\end{lstlisting}

Gaps have to be declared:

\begin{lstlisting}[language=xml]
<annotations>
    <gap-annotation set="http://ilk.uvt.nl/folia/sets/dcoi-gaps" />
</annotations>
\end{lstlisting}

\subsection{Lists}

FoLiA, like D-Coi, allows lists to be explicitly marked as shown in the following example:

\begin{lstlisting}[language=xml]
 <head><t>My grocery list</t></head>
 <list xml:id="example.list.1">
   <item xml:id="example.list.1.item.1" n="A"><t>Apples</t></item>
   <item xml:id=”example.list.1.item.2 n="B”><t>Pears</t></item>
 </list>
\end{lstlisting}

The item element may hold sentences \texttt{(s)} and words \texttt{(w)}. The D-Coi format had a \texttt{label} element, this is deprecated in favour of the \texttt{n} attribute in the item itself.

Lists, like paragraphs, sentences and headers are content elements that need not be declared and are not associated with a set or class.

\subsection{Figures}

Even figures can be encoded in the FoLiA format, although the actual figure itself can only be included as a mere reference to an external image file, but including such a reference (\texttt{src} attribute) is optional.

\begin{lstlisting}[language=xml]
 <figure xml:id="example.figure.1" n="1" src="/path/or/url/to/image/file">
   <desc>A textual description of the figure (Like ALT in HTML)</desc>
   <caption><t>The caption for the figure</t></caption>
 </figure>
\end{lstlisting}

Figures are not declared. The \texttt{caption} element may hold sentences \texttt{(s)} and words \texttt{(w)}.

\subsection{Tables}

\begin{devnotes}
There is no provision yet for encoding tables. This may be added later.
\end{devnotes}

\section{Token Annotation}

Token annotations are annotations that are placed within the word (\texttt{w}) element. They all can take any of the attributes described in section \ref{sec:paradigm}, this has to be kept in mind when reading this section. Moreover, all token annotations depend on the document being tokenised, i.e. there being a \texttt{token-annotation} declaration and \texttt{w} elements. The declaration can be as in the following example:

\begin{lstlisting}[language=xml]
<annotations>
    <token-annotation set="http://ilk.uvt.nl/folia/sets/ucto-tokconfig-nl"
      annotator="ucto" annotatortype="auto" />
</annotations>
\end{lstlisting}

Being part of a set, this implies that tokens themselves \emph{may} be assigned a class, as is for example done by the tokeniser \emph{ucto}:

\begin{lstlisting}[language=xml]
<s xml:id="example.p.1.s.1">
    <w xml:id="example.p.1.s.1.w.1" class="WORD"><t>I</t></w>
    <w xml:id="example.p.1.s.1.w.2" class="WORD"><t>see</t></w>
    <w xml:id="example.p.1.s.1.w.3" class="NUMBER"><t>2</t></w>
    <w xml:id="example.p.1.s.1.w.4" class="WORD" space="no"><t>children</t></w>
    <w xml:id="example.p.1.s.1.w.5" class="PUNCTUATION"><t>.</t></w>
</s>
\end{lstlisting}        


\subsection{Part of Speech Annotation}

The following example illustrates a simple Part-of-Speech annotation for the Dutch word ``boot'':

\begin{lstlisting}[language=xml]
<w xml:id="example.p.1.s.1.w.2">
    <t>boot</t>
    <pos class="N" />
</w>
\end{lstlisting}



However, for some tagsets simple part-of-speech annotation is not enough; there may for example be features associated with the part of speech tag. We will into this later, in section~\ref{sec:advparadigm}.

Whenever Part-of-Speech annotations are used, they should be declared in the \texttt{annotations} block as follows, the set you use may differ and all attributes are optional. In the declaration example here it is as if the annotations were made by the software \emph{Frog}. Do note the requirement of a \texttt{token-annotation} as well.

\begin{lstlisting}[language=xml]
<annotations>
    <token-annotation set="http://ilk.uvt.nl/folia/sets/ucto-tokconfig-nl" 
     annotator="ucto" annotatortype="auto" />
    <pos-annotation set="http://ilk.uvt.nl/folia/sets/CGN" 
     annotator="Frog" annotatortype="auto" />
</annotations>
\end{lstlisting}

As mentioned earlier, the declaration only sets defaults. They can be overridden in the \texttt{pos} element itself (or any other token annotation element for that matter).

\subsection{Lemma Annotation}

In the FoLiA paradigm, lemmas are perceived as classes within the (possibly open) set of all possible lemmas. Their annotation is thus as follows:

\begin{lstlisting}[language=xml]
<w xml:id="example.p.1.s.1.w.2">
    <t>boot</t>
    <lemma class="boot" />
</w>
\end{lstlisting}

And the example declaration:

\begin{lstlisting}[language=xml]
<annotations>
    <token-annotation set="http://ilk.uvt.nl/folia/sets/ucto-tokconfig-nl" 
     annotator="ucto" annotatortype="auto" />
    <lemma-annotation set="http://ilk.uvt.nl/folia/sets/mblem-nl"
     annotator="Frog" annotatortype="auto" />
</annotations>
\end{lstlisting}

\subsection{Phonetic Annotation}

Phonetic annotations can be included as follows. Similarly to lemmas, they may often refer to a set with possible open classes.

\begin{lstlisting}[language=xml]
<w xml:id="example.p.1.s.1.w.2">
    <t>boot</t>
    <phon-annotation set="ipa" class="bu:t" />
</w>
\end{lstlisting}

This is an extended token annotation element that can also be used directly on a sentence or paragraph level.

And the example declaration:

\begin{lstlisting}[language=xml]
<annotations>
    <token-annotation set="http://ilk.uvt.nl/folia/sets/ucto-tokconfig-nl" 
     annotator="ucto" annotatortype="auto" />
    <phon-annotation />
</annotations>
\end{lstlisting}

\subsection{Lexical Semantic Sense Annotation}

In semantic sense annotation, the classes in most sets will be a kind of lexical unit ID. In systems that make a distinction between lexical units and synonym sets (synsets), the synset attribute is available for notation of the latter. In systems with only synsets and no other primary form of lexical unit, the class can simply be set to the synset.

A human readable description for the \emph{sense} element, ``beeldhouwwerk'', can beplaced inside a \texttt{desc} element, but this is optional.

\begin{lstlisting}[language=xml]
<w xml:id="example.p.1.s.1.w.2">
    <t>beeld</t>
    <sense class="r_n-6220" synset="d_n-32683"><desc>beeldhouwwerk</desc></sense>
</w>
\end{lstlisting}

The example declaration is as follows:

\begin{lstlisting}[language=xml]
<annotations>
    <token-annotation set="http://ilk.uvt.nl/folia/sets/ucto-tokconfig-nl"
     annotator="ucto" annotatortype="auto" />
    <sense-annotation set="http://ilk.uvt.nl/folia/sets/cornetto" />
</annotations>
\end{lstlisting}

\subsection{Domain Tags}

This is an extended token annotation element, which means it can also be used directly in any of the content elements, such as sentence (\texttt{s}) and  paragraph (\texttt{p}). It can even be used in the \texttt{text} element itself. This annotation defines the domain of the token of content element. Example:

\begin{lstlisting}[language=xml]
<w xml:id="example.p.1.s.1.w.2">
    <t>boot</t>
    <domain class="naut"><desc>Nautical</desc></domain>
</w>
\end{lstlisting}

The value of the element may optionally be set to a human-readable label for the domain.

The declaration:

\begin{lstlisting}[language=xml]
<annotations>
    <token-annotation set="http://ilk.uvt.nl/folia/sets/ucto-tokconfig-nl"
     annotator="ucto" annotatortype="auto" />
    <domain-annotation set="http://ilk.uvt.nl/folia/sets/domains-nl" />
</annotations>
\end{lstlisting}

\subsection{Corrections}

Corrections, including but not limited to spelling corrections, can be annotated using the \texttt{correction} element. It can be applied as an extended token annotation element as in the following example, which shows a spelling correction of the misspelled word ``treee'' to its corrected form ``tree''.


\begin{lstlisting}[language=xml]
<w xml:id="example.p.1.s.1.w.1">
    <t>tree</t>
    <correction xml:id="TEST-000000001.p.1.s.1.w.1.c.1" class="spelling">
        <new>
            <t>tree</t>
        </new>
        <original>
            <t>treee</t>
        </original>
    </correction>
</w>
\end{lstlisting}

The class indicates the kind of correction, according to the set used. The \texttt{new} elements holds the actual content of the correction. The \texttt{original} element holds the content prior to correction. Note that all corrections must carry an identifier, consisting of the ID of the token with a \emph{c}, a period and a sequence number appended. In this example, what we are correcting is the actual textual content, the text element (\texttt{t}).

Whilst it may seem redundant to specify the corrected token content both under the word element (\texttt{w}), and the \texttt{new} element, and to list the original so verbosely rather than in a mere attribute, there is a good reason for this: corrections can be nested and we want to retain a full back-log. The following example illustrates the word `treee` that has been first mis-corrected to ``three'' and subsequently corrected again to ``tree'':

\begin{lstlisting}[language=xml]
<w xml:id="example.p.1.s.1.w.1">
    <t>tree</t>
    <correction xml:id="TEST-000000001.p.1.s.1.w.1.c.2" class="spelling" 
      annotator="Jane Doe" annotatortype="manual" confidence="1.0">
        <new>
            <t>tree</t>
        </new>
        <original>
            <correction xml:id="TEST-000000001.p.1.s.1.w.1.c.1" class="spelling"
               annotator="John Doe" annotatortype="manual" confidence="0.6">
                <new>
                    <t>three</t>
                </new>
                <original>
                    <t>treee</t>
                </original>
            </correction>
        </original>
    </correction>
</w>
\end{lstlisting}

In the examples above what we corrected was the actual textual content (\texttt{t}). It is however also possible to correct other annotations:
The next example corrects a part-of-speech tag; in such cases, there is no \texttt{t} element in the correction, but simply another token annotation element, or group thereof.

\begin{lstlisting}[language=xml]
<w xml:id="example.p.1.s.1.w.1">
    <t>tree</t>
    <pos class="n" />
    <correction xml:id="TEST-000000001.p.1.s.1.w.1.c.1">
        <new>
            <pos class="n" />
        </new>
        <original>
            <pos class="v" />
        </original>
    </correction>
    
</w>    
\end{lstlisting}

Again, there is a small level of necessary redundancy; the corrected element is within the \texttt{correction/new} element as well as the \texttt{w} element. Furthermore, if these two \texttt{pos} elements would differ, the FoLiA notation would be invalid.

\subsubsection{Error detection} 

The correction of an error implies the detection of an error. In some cases, detection comes without correction, for instance when the generation of correction suggestions is postponed to a later processing stage. The \texttt{errordetection} element is a very simple element that serves this purpose. It signals the existance of errors, or absence thereof:

\begin{lstlisting}[language=xml]
<w xml:id="example.p.1.s.1.w.1">
    <t>treee</t>
    <errordetection class="spelling" annotator="errorlistX" error="yes" />
</w>    
\end{lstlisting}


The \texttt{error} attribute is set to ``yes'' (which is the default value), and thus marks this as an error of class ``spelling''. We can also imagine it specifically marking something as \emph{not} being an error (in which case class is always redundant), for example due to the occurence of the word according to a lexicon:

\begin{lstlisting}[language=xml]
<w xml:id="example.p.1.s.1.w.1">
    <t>tree</t>
    <errordetection annotator="lexiconX" error="no" />
</w>    
\end{lstlisting}

Once a correction is made \emph{on the basis of} one or more kinds of error detection, the \texttt{correction} element directly embeds the \texttt{errordetection} element:

\begin{lstlisting}[language=xml]
<w xml:id="example.p.1.s.1.w.1">
    <t>tree</t>
    <correction class="spelling" annotator="John Doe">
        <new>
            <t>tree</t>
        </new>
        <original>
            <t>treee</t>
        </original>
        <errordetection class="spelling" annotator="errorlist" annotatortype="auto" error="yes" />
    </correction>
    <alt xml:id="example.p.1.s.1.w.1.alt.1">
      <correction class="spelling" annotator="errorlist" confidence="0.4">
        <new>
            <t>three</t>
        </new>
        <original>
            <t>treee</t>
        </original>
        <errordetection class="spelling" annotator="errorlist" annotatortype="auto" error="yes" />
      </correction>    
     </alt>
     <alt xml:id="example.p.1.s.1.w.1.alt.2">
      <correction class="spelling" annotator="errorlist" confidence="0.6">
        <new>
            <t>tree</t>
        </new>
        <original>
            <t>treee</t>
        </original>
        <errordetection class="spelling" annotator="errorlist" annotatortype="auto" error="yes" />
      </correction>    
    </alt>
</w>    
\end{lstlisting}

Interpret the above example as follows: An error was detected by an errorlist script. The errorlist provided two possible corrections, which are registered as \emph{alternatives} as at this stage no final decision was made to apply one. Alternatives will be explained in detail in section~\ref{sec:alternatives}.The final correction was left to a human annotator named John Doe, who made the actual correction on the basis of this error detection.

Like everything, corrections and error detection have to be declared, and have to be declared seperately. Nothing stops you from pointing them both to the same set however:

\begin{lstlisting}[language=xml]
<annotations>
    <token-annotation set="http://ilk.uvt.nl/folia/sets/ucto-tokconfig-nl" 
     annotator="ucto" annotatortype="auto" />
    <errordetection-annotation set="http://ilk.uvt.nl/folia/sets/corrections" />
    <correction-annotation set="http://ilk.uvt.nl/folia/sets/corrections" />
</annotations>
\end{lstlisting}

\subsubsection{Merges, Splits and Swaps} 

Sometimes, one wants to merge multiple tokens into one single new token, or the other way around; split one token into multiple new ones. The FoLiA format does not allow you to simply create new tokens and reassign identifiers. Identifiers are by definition permanent and should never change, as this would break backward compatibility. So such a change is therefore by definition a correction, and one uses the \texttt{correction} tag to merge and split tokens.

We will first demonstrate a merge of two tokens (``on line'') into one (``online''), the original tokens are always retained as \texttt{w-original} elements. First a peek at the XML prior to merging:

\begin{lstlisting}[language=xml]
<s xml:id="example.p.1.s.1">
    <w xml:id="example.p.1.s.1.w.1">
        <t>on</t>
    </w>
    <w xml:id="example.p.1.s.1.w.2">
        <t>line</t>
    </w>                       
</s>  
\end{lstlisting}

And after merging:

\begin{lstlisting}[language=xml]
<s xml:id="example.p.1.s.1">
 <correction xml:id="example.p.1.s.1.c.1" class="merge">
    <new>
        <w xml:id="example.p.1.s.1.w.1-2">        
            <t>online</t>
        </w>
    </new>
    <original>
        <w xml:id="example.p.1.s.1.w.1">
            <t>on</t>
        </w>
        <w xml:id="example.p.1.s.1.w.2">
            <t>line</t>
        </w>                         
    </original>
 </correction>               
</s>
\end{lstlisting} 

Note that the correction element, being a kind of extended token annotation, is here a member of the sentence (\texttt{s}), rather than the word token (\texttt{w}) as in all previous examples. The new identifier denotes the span of the merge, separated by a hyphen, so we get \texttt{.w.1-2} if we merge from \texttt{.w.1} to \texttt{.w.2}.

Now we will look at a split, the reverse of the above situation. Prior to splitting, assume we have:

\begin{lstlisting}[language=xml]
<s xml:id="example.p.1.s.1">
 <w xml:id="example.p.1.s.1.w.1">
    <t>online</t>
 </w>                         
</s>
\end{lstlisting}

After splitting:

\begin{lstlisting}[language=xml]

<s xml:id="example.p.1.s.1">
 <correction xml:id="example.p.1.s.1.c.1" class="split">
    <new>    
        <w xml:id="example.p.1.s.1.w.1_1">
            <t>on</t>
        </w>
        <w xml:id="example.p.1.s.1.w.1_2">
            <t>line</t>
        </w>                        
    </new>
    <original>
        <w xml:id="example.p.1.s.1.w.1">
            <t>online</t>
        </w>
    </original>
 </correction>               
</s>
\end{lstlisting}

The new identifiers represent the index of the new tokens, separated by a underscore, so given \texttt{.w.1}  we get \texttt{.w.1\_1} for the first split result, \texttt{.w.1\_2} for the second, and so on...

The same principle as used for merges and splits can also be used for performing ``swap'' corrections:

\begin{lstlisting}[language=xml]

<s xml:id="example.p.1.s.1">
 <correction xml:id="example.p.1.s.1.c.1" class="split">
    <new>    
        <w xml:id="example.p.1.s.1.w.2">
            <t>on</t>
        </w>
        <w xml:id="example.p.1.s.1.w.1">
            <t>line</t>
        </w>
    </new>
    <original>
        <w xml:id="example.p.1.s.1.w.1">
            <t>line</t>
        </w>
        <w xml:id="example.p.1.s.1.w.2">
            <t>on</t>
        </w>
    </original>
 </correction>               
</s>
\end{lstlisting}

Note that in such a swap situation, the identifiers of the word tokens will appear out of sequence after correction, due to the principle that identifiers never change once set.


\subsubsection{Omissions and Insertions}

Omissions, words that are removed in correction, and insertions, words inserted during correction, are dealt with in a way similar to merges, splits and swaps. For omissions, the \texttt{new} element is simply empty. In the following example the word ``the'' was duplicated and removed in correction:

\begin{lstlisting}[language=xml]
<s xml:id="example.p.1.s.1">
 <w xml:id="example.p.1.s.1.w.1">
    <t>the</t>
 </w>
 <correction xml:id="example.p.1.s.1.c.1" class="duplicate">
    <new>                        
    </new>
    <original>
        <w xml:id="example.p.1.s.1.w.2">
            <t>the</t>
        </w>
    </original>
 </correction>  
 <w xml:id="example.p.1.s.1.w.3">
    <t>man</t>
 </w>
</s>
\end{lstlisting}

For insertions, the \texttt{original} element is empty. 

\begin{lstlisting}[language=xml]
<s xml:id="example.p.1.s.1">
 <w xml:id="example.p.1.s.1.w.1">
    <t>the</t>
 </w>
 <correction xml:id="example.p.1.s.1.c.1" class="duplicate">
    <new>                            
        <w xml:id="example.p.1.s.1.w.1_1">
            <t>old</t>
        </w>
    </new>
    <original>
    </original>
 </correction>  
 <w xml:id="example.p.1.s.1.w.2">
    <t>man</t>
 </w>
</s>
\end{lstlisting}


\subsubsection{Correction prior to tokenisation} 

There is another special use of the correction element. Sometimes corrections or normalisations occur prior to tokenisation, think for example about correcting OCR-errors. To accommodate this, the \texttt{correction} element can be used inline within the text content element (\texttt{t}) of a paragraph or sentence, which is by definition untokenised.

Without correction:

\begin{lstlisting}[language=xml]
<s xml:id="example.p.1.s.1.w.1">
    <t>Look at thi.s untokenised sentence.</t>
</s>            
\end{lstlisting}

With correction:

\begin{lstlisting}[language=xml]
<s xml:id="example.p.1.s.1">
    <t corrections="yes">Look at <correction xml:id="example.p.1.s.1.c.1"
     class="ocrcorrection">
     <new>
        <t>this</t>
     </new>
     <original>
       <t>thi.s</t>
     </original></correction> untokenised sentence.
    </t>
</s>                         
\end{lstlisting}

Although correction is used inline here, rather than as a normal token annotation, its usage is still identical. For clarity's sake, the class of course depends on the set and is as always never predefined in FoLiA itself.

Note that the text element gains an extra mandatory attribute, \texttt{correction} with value \emph{yes} (default if unspecified is no), which signals that there are inline corrections \emph{within} the text element. This is to make the job of parsers easier.

\subsection{Morphological Analysis}

\begin{devnotes}
Still to be done.. The \texttt{morphemes} and \texttt{morpheme} elements will be reserved for this.
\end{devnotes}


\section{Alternative Token Annotations}
\label{sec:alternatives}

The FoLiA format does not just allow for a single ``favoured'' annotation per token, in addition it allows for the recording of alternative annotations. Alternative token annotations are grouped within one or more \texttt{alt} elements. If multiple annotations are grouped together under the same \texttt{alt} element, then they are deemed dependent and form a single set of alternatives.

Each alternative has a unique identifier, formed in the already familiar fashion. In the following example we see the Dutch word ``bank'' in the sense of a sofa, alternatively we see two alternative annotations with a different sense and domain.

\begin{lstlisting}[language=xml]
<w xml:id="example.p.1.s.1.w.1">
    <t>bank</t>
    <domain class="furniture" />
    <sense class="r_n-5918" synset="d_n-21410" 
     annotator="John Doe" annotatortype="manual" 
     confidence="1.0">zitmeubel</sense>
    <alt xml:id="example.p.1.s.1.w.1.alt.1">
        <domain class="finance" />
        <sense class="r_n-5919" synset="d_n-27025"
         annotator="Jane Doe" annotatortype="manual" 
         confidence="0.6">geldverlenende instelling</sense>        
    </alt>
    <alt xml:id="example.p.1.s.1.w.1.alt.2">
        <domain class="geology" />
        <sense class="r_n-5920" synset="d_n-38257"
         annotator="Jim Doe" annotatortype="manual"
         confidence="0.1">zandbank</sense>        
    </alt>    
</w>
\end{lstlisting}


\section{Span Annotation}

Not all annotations can be realised as token annotations. Some typically span multiple tokens. For these we introduce a kind of offset notation in separate \emph{annotation layers}. These annotation layers are embedded at the sentence level, \emph{after} the word tokens. Within these layers, references are made to these word tokens. Each annotation layer is specific to a kind of span annotation.

The layer elements themselves may also take the \texttt{set}, \texttt{annotator}, \texttt{annotatortype}, or \texttt{confidence} attributes. Which introduces the defaults for all the span annotations under it. They in turn may of course always chose to override this.

\subsection{Entities}

Named entities or other multi-word units can be encoded in the \texttt{entities} layer. Below is an example of a full sentence in which one name is tagged. Each entity should have a unique identifier.


\begin{lstlisting}[language=xml]
<s xml:id="example.p.1.s.1">
  <t>The Dalai Lama greeted him.</t>
  <w xml:id="example.p.1.s.1.w.1"><t>The</t></w>
  <w xml:id="example.p.1.s.1.w.2"><t>Dalai</t></w>
  <w xml:id="example.p.1.s.1.w.3"><t>Lama</t></w>
  <w xml:id="example.p.1.s.1.w.4"><t>greeted</t></w>
  <w xml:id="example.p.1.s.1.w.5"><t>him</t></w>
  <w xml:id="example.p.1.s.1.w.6"><t>.</t></w>
  <entities>
    <entity xml:id="example.p.1.s.1.entity.1" class="person">
        <wref xml:id="example.p.1.s.1.w.2" />
        <wref xml:id="example.p.1.s.1.w.3" />
    </entity>
  </entities>
</s>
\end{lstlisting}

Note that elements that are not part of any span annotation need never be included in the layer.

\subsection{Syntax}

A very typical form of span annotation is syntax annotation. This is done within the \texttt{syntax} layer and introduces a nested hierarchy of syntactic unit (\texttt{su}) elements. Each syntactic unit should have a unique identifier.

\begin{lstlisting}[language=xml]
<s xml:id="example.p.1.s.1">
  <t>The Dalai Lama greeted him.</t>
  <w xml:id="example.p.1.s.1.w.1"><t>The</t></w>
  <w xml:id="example.p.1.s.1.w.2"><t>Dalai</t></w>
  <w xml:id="example.p.1.s.1.w.3"><t>Lama</t></w>
  <w xml:id="example.p.1.s.1.w.4"><t>greeted</t></w>
  <w xml:id="example.p.1.s.1.w.5"><t>him</t></w>
  <w xml:id="example.p.1.s.1.w.6"><t>.</t></w>
  <syntax>
    <su xml:id="example.p.1.s.1.su.1" class="s">     
      <su xml:id="example.p.1.s.1.su.1_1" class="np">
          <su xml:id="example.p.1.s.1.su.1_1_1" class="det">
             <wref xml:id="example.p.1.s.1.w.1" />       
          </su>
          <su xml:id="example.p.1.s.1.su.1_1_2" class="pn">
             <wref xml:id="example.p.1.s.1.w.2" />
             <wref xml:id="example.p.1.s.1.w.3" />        
          </su>         
       </su>
     </su>
     <su xml:id="example.p.1.s.1.su.1_2" class="vp"> 
        <su xml:id="example.p.1.s.1.su.1_1_1" class="v">
            <wref xml:id="example.p.1.s.1.w.4" />       
        </su>
        <su xml:id="example.p.1.s.1.su.1_1_2" class="pron">
          <wref xml:id="example.p.1.s.1.w.5" />       
        </su>
     </su>    
    </su>
  </syntax>
</s>
\end{lstlisting}

Just to prevent any misunderstanding, the classes depend on the set used, so you can use whatever system of syntactic annotation you desire. Moreover, any of the \texttt{su} elements can have the common attributes \texttt{annotator}, \texttt{annotatortype} and \texttt{confidence}.


The above example illustrated a fairly simple syntactic parse. Dependency parses are possible too. Dependencies are listed separate from the syntax in an extra annotation layer, as shall be explained in the next section.

The declaration is as follows:

\begin{lstlisting}[language=xml]
<annotations>
    <token-annotation set="http://ilk.uvt.nl/folia/sets/ucto-tokconfig-nl" 
     annotator="ucto" annotatortype="auto" />
    <syntax-annotation set="http://ilk.uvt.nl/folia/sets/syntax-nl" />
</annotations>
\end{lstlisting}

\subsection{Dependency Relations}

Dependency relations are relations between syntactic units (or spans of tokens). This relation is often of a particular class and consists of a head component and a dependent component. In the sample ``He sees'', there is  syntactic dependency between the two words: ``sees'' is the head, and ``He'' is the dependant, and the relation class is something like ``subject'', as the dependant is the subject of the head word. Each dependency relation is explicitly noted.

The element \texttt{dependencies} introduces this annotation layer. Within it, \texttt{dependency} elements describe all dependency pairs. 

In the below example, we show a Dutch sentence parsed with the Alpino Parser \cite{ALPINO}. We show not only the dependency layer, but also the syntax layer. The \texttt{dependency} element always contains one head element (\texttt{hd}) and one dependant element (\texttt{dep}), both can refer to a syntactic unit by means of the \texttt{su} attribute. Additionally, the words they cover are reiterated in the usual fashion. For a better understanding, the figure below illustrates the syntactic parse with the dependency relations.

\begin{figure}[h]
\begin{center}
\includegraphics[width=60.0mm]{alpino.png}
\end{center}
\caption{Alpino dependency parse for the Dutch sentence ``De man begroette hem.''}
\label{fig:arch} 
\end{figure}
\FloatBarrier

\begin{lstlisting}[language=xml]
<s xml:id="example.p.1.s.1">
  <t>De man begroette hem.</t>
  <w xml:id="example.p.1.s.1.w.1"><t>De</t></w>
  <w xml:id="example.p.1.s.1.w.2"><t>man</t></w>
  <w xml:id="example.p.1.s.1.w.3"><t>begroette</t></w>
  <w xml:id="example.p.1.s.1.w.4"><t>hem</t></w>
  <w xml:id="example.p.1.s.1.w.5"><t>.</t></w>
  <syntax set="http://ilk.uvt.nl/folia/sets/alpino">
    <su xml:id="example.p.1.s.1.su.1" class="top">     
        <su xml:id="example.p.1.s.1.su.1_1" class="smain">     
            <su xml:id="example.p.1.s.1.su.1_1_1" class="np">     
                <su xml:id="example.p.1.s.1.su.1_1_1_1" class="top">     
                    <wref xml:id="example.p.1.s.1.w.1" />       
                </su>
                <su xml:id="example.p.1.s.1.su.1_1_1_2" class="top">     
                    <wref xml:id="example.p.1.s.1.w.2" />
                </su> 
            </su>
            <su xml:id="example.p.1.s.1.su.1_1_2" class="verb">     
                <wref xml:id="example.p.1.s.1.w.3" />   
            </su>
            <su xml:id="example.p.1.s.1.su.1_1_3" class="pron">     
                <wref xml:id="example.p.1.s.1.w.4" />   
            </su>
        </su>
        <su xml:id="example.p.1.s.1.su.1_2" class="punct">
            <wref xml:id="example.p.1.s.1.w.5" />               
        </su> 
    </su>
  </syntax>
  <dependencies>
    <dependency xml:id="example.p.1.s.1.dependency.1" class="det">
        <hd su="example.p.1.s.1.su.1_1_1_2">
           <wref xml:id="example.p.1.s.1.w.2" />   
        </hd>
        <dep ref="example.p.1.s.1.su.1_1_1_1">
            <wref xml:id="example.p.1.s.1.w.1" />   
        </dep>
    </dependency>
    <dependency xml:id="example.p.1.s.1.dependency.2" class="obj1">
        <hd su="example.p.1.s.1.su.1_1_2">
            <wref xml:id="example.p.1.s.1.w.3">
        </hd>
        <dep su="example.p.1.s.1.su.1_1_3">
            <wref xml:id="example.p.1.s.1.w.4" />   
        </dep>
    </dependency>
  </dependencies>
</s>
\end{lstlisting}

The declaration:

\begin{lstlisting}[language=xml]
<annotations>
    <token-annotation set="http://ilk.uvt.nl/folia/sets/ucto-tokconfig-nl" 
     annotator="ucto" annotatortype="auto" />
    <syntax-annotation set="http://ilk.uvt.nl/folia/sets/alpino-syntax" /> 
    <dependency-annotation set="http://ilk.uvt.nl/folia/sets/alpino-dep" />
</annotations>
\end{lstlisting}

\subsection{Chunking}

Unlike a full syntactic parse, chunking is not nested. The layer for this type of linguistic annotation is predictably called \texttt{chunking}. The span annotation element itself is \texttt{chunk}.

\begin{lstlisting}[language=xml]
<s xml:id="example.p.1.s.1">
  <t>The Dalai Lama greeted him.</t>
  <w xml:id="example.p.1.s.1.w.1"><t>The</t></w>
  <w xml:id="example.p.1.s.1.w.2"><t>Dalai</t></w>
  <w xml:id="example.p.1.s.1.w.3"><t>Lama</t></w>
  <w xml:id="example.p.1.s.1.w.4"><t>greeted</t></w>
  <w xml:id="example.p.1.s.1.w.5"><t>him</t></w>
  <w xml:id="example.p.1.s.1.w.6"><t>.</t></w>
  <chunking>
    <chunk xml:id="example.p.1.s.1.chunk.1">       
        <wref xml:id="example.p.1.s.1.w.1" />       
        <wref xml:id="example.p.1.s.1.w.2" />       
        <wref xml:id="example.p.1.s.1.w.3" />        
    </chunk>
    <chunk xml:id="example.p.1.s.1.chunk.2">       
        <wref xml:id="example.p.1.s.1.w.4" />
    </chunk>
    <chunk xml:id="example.p.1.s.1.chunk.3">       
        <wref xml:id="example.p.1.s.1.w.5" />
        <wref xml:id="example.p.1.s.1.w.6" />
    </chunk>    
  </chunking>
</s>
\end{lstlisting}


The declaration:

\begin{lstlisting}[language=xml]
<annotations>
    <chunking-annotation set="http://ilk.uvt.nl/folia/sets/syntax-nl" />
</annotations>
\end{lstlisting}



\subsection{Semantic roles}

\begin{devnotes}
Still to be done.. The \texttt{semroles} layer and \texttt{semrole} span annotation element will be reserved for this.
\end{devnotes}

\subsection{Alterative Span Annotations}

With token annotations one could specify an unbounded number of alternative annotations. This is possible for span annotations as well, but due to the different nature of span annotations this happens in a slightly different way.

Where we used \texttt{alt} for token annotations, we now use \texttt{altlayers} for span annotations. Under this element several alternative layers can be presented. Analogous to \texttt{alt}, any layers grouped together are assumed to be somehow dependent. Multiple \texttt{altlayers} can be added to introduce independent alternatives. Each alternative should be associated with a unique identifier, which uses ``alt'' rather than ``altlayers''. 

Below is an example of a sentence that is chunked in two ways:

\begin{lstlisting}[language=xml]
<s xml:id="example.p.1.s.1">
  <t>The Dalai Lama greeted him.</t>
  <w xml:id="example.p.1.s.1.w.1"><t>The</t></w>
  <w xml:id="example.p.1.s.1.w.2"><t>Dalai</t></w>
  <w xml:id="example.p.1.s.1.w.3"><t>Lama</t></w>
  <w xml:id="example.p.1.s.1.w.4"><t>greeted</t></w>
  <w xml:id="example.p.1.s.1.w.5"><t>him</t></w>
  <w xml:id="example.p.1.s.1.w.6"><t>.</t></w>
  <chunking>
    <chunk xml:id="example.p.1.s.1.chunk.1">       
        <wref xml:id="example.p.1.s.1.w.1" />       
        <wref xml:id="example.p.1.s.1.w.2" />       
        <wref xml:id="example.p.1.s.1.w.3" />        
    </chunk>
    <chunk xml:id="example.p.1.s.1.chunk.2">       
        <wref xml:id="example.p.1.s.1.w.4" />
    </chunk>
    <chunk xml:id="example.p.1.s.1.chunk.3">       
        <wref xml:id="example.p.1.s.1.w.5" />
        <wref xml:id="example.p.1.s.1.w.6" />
    </chunk>    
  </chunking>
  <altlayers xml:id="example.p.1.s.1.alt.1">
       <chunking annotator="John Doe" 
        annotatortype="manual" confidence="0.0001">
        <chunk xml:id="example.p.1.s.1.alt.1.chunk.1">       
            <wref xml:id="example.p.1.s.1.w.1" />       
            <wref xml:id="example.p.1.s.1.w.2" />                       
        </chunk>
        <chunk xml:id="example.p.1.s.1.alt.1.chunk.2">       
            <wref xml:id="example.p.1.s.1.w.3" />  
            <wref xml:id="example.p.1.s.1.w.4" />
        </chunk>
        <chunk xml:id="example.p.1.s.1.alt.1.chunk.3">       
            <wref xml:id="example.p.1.s.1.w.5" />
            <wref xml:id="example.p.1.s.1.w.6" />
        </chunk>    
      </chunking>   
  </altlayers>
</s>
\end{lstlisting}

The support for alternatives and the fact that multiple layers (including those of different types) cannot be nested in a single inline structure, should make clear why FoLiA uses a stand-off notation alongside an inline notation. 


\section{Advanced Paradigm}
\label{sec:advparadigm}

We introduced the FoLiA paradigm in section~\ref{sec:paradigm}. Now we will introduce some of the more advanced aspects of the FoLiA paradigm. These are relevant especially if you seek to extend FoLiA with annotations not yet supported.

\subsection{Human readable Descriptions}

Any token annotation element or span annotation element may hold a \texttt{desc} element containing a human readable description for the annotation. An example of this has been already shown for the \texttt{sense} and \texttt{gap} elements.

\subsection{Subsets}

For some annotations, associating a single class from a set is not sufficient. FoLiA provides the option to associate multiple classes, or to associate classes with specific subsets. 

Take a look at the following abstract case: \texttt{X} here is a \emph{fictional} token/span annotation. It is associated with two classes. The fictional element \texttt{Y} belongs with \texttt{X}, and is custom to the specific annotation.

\begin{lstlisting}[language=xml]
<X set="a">
    <Y class="p">
    <Y class="q">
</X>
\end{lstlisting}

The classes can be made members of a certain subset, rather than the main set. The subsets are assumed to in turn defined by the main set. No special declarations will be necessary.

\begin{lstlisting}[language=xml]
<X set="a">
    <Y subset="f" class="p">
    <Y subset="g" class="q">
</X>
\end{lstlisting}

Optionally, \texttt{X} itself can also be associated with a class, the class in this case would have to represent all sub-parts.

\subsection{Part-of-Speech tags with features}
\label{sec:posfeat}

Part-of-speech tags are a good example of the scenario outlined above. Part-of-speech tags may consist of multiple features, which in turn \emph{may} be associated with specific subsets:

\begin{lstlisting}[language=xml]
<w xml:id="example.p.1.s.1.w.2">
    <t>boot</t>
    <pos class="N(soort,ev,basis,zijd,stan)">
        <desc>Noun, singular, neuter</desc>
        <posfeat subset="head" class="N" />
        <posfeat subset="ntype" class="soort" />
        <posfeat subset="number" class="ev" />
        <posfeat subset="degree" class="basis" />
        <posfeat subset="gender" class="zijd" />
        <posfeat subset="case" class="stan" />
    </pos>
</w>
\end{lstlisting}

\section{Alignments}

FoLiA provides a facility to align parts of your document with other parts of your document, or even with parts of other FoLiA documents. These are called \emph{alignments} and are implemented using either the \texttt{alignment} element, a form of extended token annotation (that may thus also be applied on other levels such as sentences or paragraphs), or \texttt{complexalignment}, a form of span annotation for more complex word alignments.

Consider the following two aligned sentences from two \emph{distinct} FoLiA documents in different languages:

\begin{lstlisting}[language=xml]
<s xml:id="example-english.p.1.s.1">
  <t>The Dalai Lama greeted him.</t>
  <alignment class="french-translation" target="doc-french.xml">
        <ref xml:id="doc-french.p.1.s.1.w.1" />
  </alignment>
</s>

<s xml:id="example-french.p.1.s.1">
  <t>Le Dalai Lama le saluait.</t>
  <alignment class="english-translation" target="doc-english.xml">
        <aref xml:id="doc-english.p.1.s.1.w.1" />
  </alignment>
</s>
\end{lstlisting}


Although the above example has a single alignment reference (\texttt{aref}), it is not forbidden to link to specify multiple references within the \texttt{alignment} block. For more complex alignments however, such as word alignment that including many-to-one, one-to-many or many-to-many alignments, the span annotation element \texttt{complexalignment} may be more suitable. The following example illustrates a many-to-many word-alignment of the word ``Dalai Lama'' from English to French. This takes places within the \texttt{complexalignments} annotation layer.

\begin{lstlisting}[language=xml]
<s xml:id="example-english.p.1.s.1">
  <t>The Dalai Lama greeted him.</t>
  <w xml:id="example-english.p.1.s.1.w.1"><t>The</t></w>
  <w xml:id="example-english.p.1.s.1.w.2"><t>Dalai</t></w>
  <w xml:id="example-english.p.1.s.1.w.3"><t>Lama</t></w>
  <w xml:id="example-english.p.1.s.1.w.4"><t>greeted</t></w>
  <w xml:id="example-english.p.1.s.1.w.5"><t>him</t></w>
  <w xml:id="example-english.p.1.s.1.w.6"><t>.</t></w>
  <complexalignments>
        ...
        <complexalignment class="french-translation" target="doc-french.xml">
            <wref xml:id="example-english.p.1.s.1.w.2">
            <wref xml:id="example-english.p.1.s.1.w.3">
            <aref xml:id="example-french.p.1.s.1.w.2">
            <aref xml:id="example-french.p.1.s.1.w.3">
        </complexalignment>
        ...
  </complexalignments>
</s>
\end{lstlisting}




\section{Metadata}

FoLiA has support for metadata, most notably the extensive and mandatory declaration section for all used annotations which you have seen throughout this documentation. To complement this, there is FoLiA's native metadata system, in which simple metadata fields can be defined and used at will. FoLiA is also able to operate with IMDI or CMDI metadata, either in external file or stored inline. Note however that storing CMDI or IMDI inside your FoLiA document may cause problems when you want to validate your FoLiA document. It is also incompatible with CMDI or IMDI editors that are unaware of FoLiA.

Reference to CMDI in external file proceeds in the following simple fashion:

\begin{lstlisting}[language=xml]
<metadata type="cmdi" src="/path/or/url/to/metadata.cmdi">
 ...
</metadata>
\end{lstlisting}

The procedure for IMDI is the same:

\begin{lstlisting}[language=xml]
<metadata type="imdi" src="/path/or/url/to/metadata.imdi">
 ...
</metadata>
\end{lstlisting}

If you use neither CMDI nor IMDI, then you can use FoLiA's native system, which is very simple: It introduces the \texttt{meta} element that allows you to define key value pairs as follows:

\begin{lstlisting}[language=xml]
<metadata type="native">
    <annotations>
    ..
    </annotations>
    <meta id="title">Title to my document</meta>
    <meta id="language">eng</meta>
</metadata>
\end{lstlisting}

You can simply define fields with custom IDs, but the following fields are pre-defined and recommended to be filled:

\begin{itemize}
\item \textbf{title} -- The title of the FoLiA document
\item \textbf{language} -- An ISO-639-3 language code identifying the language the document is 
\item \textbf{date} -- The date of publication in \texttt{YYYY-MM-DD} format
\item \textbf{publisher} -- The publishing institution or individual
\item \textbf{license} -- The type of license of the document (for example: \emph{GNU Free Documentation License})
\end{itemize}



\appendix
\chapter{Common Queries}

Considering the fact that FoLiA is an XML-based format, XPath and its derivatives are the designated tools for searching in a FoLiA document. Some common queries are listed below:

\begin{itemize}
\item XPath query for all paragraphs: \texttt{/FoLiA/text//p}
\item XPath query for all sentences: \texttt{/FoLiA/text//s[not(parent::s)}] \emph{Selects only top-level sentences, not sentences embedded within sentences (e.g quotes)}
\item XPath query for all words: \texttt{/FoLiA/text//w[not(ancestor::original)]} \emph{The conditional clause is required to correctly deal with correction syntax.}
\item XPath query for all words with lemma X: \\ \texttt{/FoLiA/text//w[not(ancestor::original)]/lemma[@class == "X"]}
\item XPath query for the text of all words: \\ \texttt{/FoLiA/text//w[not(ancestor::original)]/t/child::text()}
\end{itemize}




\bibliographystyle{plain}
\bibliography{folia}


\end{document}
